\documentclass[11pt]{article}
\usepackage[margin  = 1in]{geometry}
\usepackage{hyperref}
\bibliographystyle{ieeetr}
\usepackage{amsmath} 

\begin{document}

\title{Deep Learning Methods for Optimizing Terahertz Leaky Wave Antenna Design}
\author{J. Neronha, H. Guerboukha, and D. Mittleman \\ \\ \small School of Engineering, Brown University, Providence, RI 02912}
\date{}

\maketitle

\section*{Introduction}

The predictive power of neural networks and machine learning techniques more generally have been applied to a vast array problems ever since the neural network was proposed as a computational model for the brain in the mid-twentieth century. \cite{McCulloch:1943vq} This, of course, includes communications technology -- neural networks have been used extensively in the field because of their unique ability to approximate accurate solutions to nonlinear problems that are commonplace in antenna design. Machine learning models are particularly useful in situations where an analytical solution cannot be obtained and/or numerical simulations are expensive \cite{Kim, Massa}, for example when solving the ``direct" problem of approximating an antenna's output \cite{8608745} or modeling a metasurface \cite{Nadell:19}, which could take a finite-element model hours or longer to solve, limiting real-time simulation and design. The ``inverse" problem considers the opposite, for example reconstructing an image from scattered light \cite{Sun:18}. \\

\noindent We are particularly interested in the inverse problem of terahertz leaky-wave antenna (LWA) design because of its high applicability to the field of communications -- given some arbitrarily desired far-field pattern, how can we quickly design, fabricate, and test an antenna that meets our needs? LWAs are simple metallic waveguides that have proven very effective in the terahertz range and are particularly interesting because they emit radiation at a frequency-dependent angle with a one-to-one relationship between frequency and angle, which is quite valuable given the narrow character of beams in the terahertz range. \cite{doi:10.1063/5.0033126} They have been used in a number of diverse applications including link discovery \cite{Ghasempour:2020tz}, multiplexing and demultiplexing \cite{Karl:2015uh, Ma:2017vo}, and for radar and object detection purposes \cite{Amarasinghe:20}. Leaky-wave antennas also stand out for our purposes of rapid design and experimentation because of the ease of fabricating them using novel hot-stamping techniques \cite{Guerboukha:21}. \\

\noindent This inverse problem has been explored in the terahertz range using deep neural networks in the context of designing structures to obtain an optimized geometry that produces the desired signal, particularly in relation to metasurface design \cite{Deng:21, 9602997}. The inverse Leaky-wave antenna problem has also been explored, but using a genetic algorithm and outside of the terahertz range at much higher frequencies \cite{Jafar-Zanjani:2018vy}. As a result, in this paper, we propose a model to predict the ideal LWA geometry that will generate desired far-field radiation in the terahertz range.

\section*{Methodology}

We train a model to predict a slot geometry for a desired far-field signal. In particular, we consider a one-dimensional discretized slot where each component 0.5 micron slot can either be transparent or metallic, similar to the approach taken in past explorations of metasurface design \cite{Jafar-Zanjani:2018vy}. This design essentially creates a non-uniform periodic waveguide whose peaks can be predicted by Floquet theory. In a leaky wave guide with plate separation $h$, the dispersion constant $\beta$ for a given period $\Lambda$ and Floquet mode $p$ for a wave vector $k_0$ is given by:

\[\beta=\sqrt{k_0^2 - (\frac{\pi}{h})^2} +\frac{2\pi p}{\Lambda} \tag{1} \label{eq:special}\] 

\noindent Oftentimes, the primary objective design of antenna design is to generate peaks of specific magnitudes at specific locations. We use Eq. 1 to  obtain the location of all possible Floquet peaks for periodic leaky wave antennas, which form the set of possible peaks for which the model can design a geometry and only keep the solutions where $\beta > |k_0|$ (i.e. where radiation leaks from the antenna -- forward scattering if $k_0$ is positive, and backwards scattering if $k_0$ is negative). This is important because it ensures that users of the model cannot attempt to design a slot that will place a peak at an angle that is impossible given the dispersion relation for the waveguide. For our LWA with a plate separation of 1 mm operating at a frequency of 200 GHz, we plot the possible peaks generated by Floquet theory in Figure 1 below. \\

\noindent We build a model of a leaky-wave antenna in COMSOL Multiphysics with a slot of length 18 mm divided into 36 sub-slots, each of which is 500 microns in length. 18 sub-slots (exactly half) of the slots are randomly chosen as to be transparent (i.e. a scattering boundary layer) whereas the balance remain metallic and thus do not leak radiation. We automate this random geometry generation and simulation using MPh \cite{john_hennig_2022_6312347}, an open-source Python package that enables controlling COMSOL via its native API. Basic combinatorics theory tells us that there are over 9 billion possible slot designs that can be generated, indicating the usefulness of deep neural networks in this prediction task given the sheer number of possible outputs. \\

\noindent The COMSOL simulation data is used as the data source for our deep neural network. The network, which is built with TensorFlow, has six dense layers, each of which have 2000 neurons and use a LeakyReLU activation function with $\alpha = 0.1$. The model uses the Adam optimizer and has a learning rate of 0.0001 and predicts the output of each individual sub-slot with a sigmoid activation function indicating the probability that a given sub-slot should be transparent. The model architecture is visualized in Figure 2.

\section*{Results and Discussion}

\section*{Conclusion}

\bibliographystyle{plain}
\bibliography{references.bib}
\end{document}
